\documentclass[10pt]{article}

\oddsidemargin 0.2cm \evensidemargin 0.2cm \textwidth 16.5cm
\textheight 22.5cm \topmargin -0.46cm

\begin{document}


\begin{center}
Fundamentos de F\'{i}sica III - UFES-Alegre

1$^{\underline{a}}$ Trabalho em Grupo - Lei de Coulomb (25/08/2023, entrega em 30/08/2023)

Semestre 2023/2 - Turmas de Ci\^{e}ncia da Computa\c{c}\~{a}o e Engenharia Química
\end{center}

Nome : \dotfill 

Nome : \dotfill 

Nome : \dotfill 

\begin{enumerate}
	

\item Decide-se dividir uma carga el\'{e}trica $Q$ em duas partes, localizadas em part\'{\i}culas separadas por uma dist\^{a}ncia $d$, sendo que 
uma carga teria $q$ e outra $(Q-q)$. Para $q$ arbitr\'{a}rio (por\'{e}m de mesmo sinal que $Q$), calcule a posi\c{c}\~{a}o ao longo da
linha que une as part\'{\i}culas tal que a for\c{c}a Coulombiana seja nula (nessa posi\c{c}\~{a}o).

Qual \'{e} o valor de $q$ tal que a for\c{c}a de repuls\~{a}o\ Coulombiana entre as part\'{\i}culas seja m\'{a}xima ? Com esse $q$, qual \'{e} a 
posi\c{c}\~{a}o entre as part\'{\i}culas tal que a for\c{c}a Coulombiana seja nula (nessa posi\c{c}\~{a}o) ?

Não precisa usar nota\c{c}\~{a}o vetorial.


\item Duas part\'{i}culas, com cargas el\'{e}tricas $q_{1} = q_{2} = 3,20\times 10^{-19} C$ est\~{a}o ao longo do eixo $y$, sendo $y_1 = d = +1,0 m$ 
e $y_2 = -d = -1,0 m$. Uma terceira part\'{i}cula, com carga $q_{3} = -1,60\times 10^{-19} C$, est\'{a} situada ao longo do eixo $x$, com $x$ podendo 
variar entre $0,0\,m$ e $d = +1,0\,m$.

Para qual valor de $x$ a soma das for\c{c}as eletroest\'{a}ticas sobre $q_3$ \'{e} m\'{i}nima em m\'{o}dulo ? E m\'{a}xima ? 

Diagrame vetorialmente as duas for\c{c}as e a for\c{c}a resultante sobre $q_3$ para os dois valores de $x$ encontrados.

Use nota\c{c}\~{a}o vetorial em toda a resolu\c{c}\~{a}o e faça analiticamente, substituindo numericamente somente ao final.


\end{enumerate}

\begin{center}
Boa Sorte !

Prof. Roberto Colistete J\'{u}nior

\bigskip
\end{center}

\end{document}
